\chapter{总结和展望} \label{cha:conclusion}
\section{论文总结}
当前越来越多的应用部署在数据中心,不同的应用对数据中心的需求不同,数据中心传统策略TCP不能满足不同应用对网络资源的需求。
基于此,本文从流级别和任务级别对数据中心的传输进行优化。
本文建立了基于ECN标记的流传输模型,通过此模型,可以DCTCP等策略建模,并分析策略参数;
提出了期限敏感流的负载自适应传输优化方案,即使在网络拥塞程度很严重的情形下依然可以根据应用的期限和网络拥塞程度对传输进行优化;
提出了基于流持续时间的传输优化方案,在无需得知流大小的前提下,可以同时优化有期限的数据流和流完成时间。
提出了数据中心任务级传输优化模型,根据任务级传输优化模型,提出了数据中心传输任务调度的2-近似算法,并证明其近似度。
提出了基于流信息的任务级传输优化方案,在预先得知流信息的前提下,对传输任务进行优化。
提出了基于重要性和网络拥塞的任务传输调度方案,可以根据应用的重要性和网络拥塞程度对传输任务进行调度。
最后,设计并且实现了数据中心传输系统FlyTransfer,并对之进行性能评估。

本文的主要结论如下:

\textbf{1 提出了基于ECN标记的流传输模型,对基于ECN标记的流传输方案进行建模}

本文提出了基于ECN标记的流传输模型。
在基于ECN标记的流传输策略中,当网络出现拥塞时,
交换机会给数据包标记CE,当接收端收到被标记的数据包时,对此数据包回复的ACK会标记ECN。
发送端根据被标记的ECN比例进行拥塞窗口计算。
使用基于ECN标记的的流传输模型,可以对基于ECN标记的流传输模型进行建模分析。



\textbf{2 提出了期限敏感流的负载自适应传输优化方案,当网络拥塞严重时依然能根据期限和网络情形进行速率控制}

本文提出了在设计数据流传输方案时,
应该遵循一个简单的原则:
拥有不同截止期限(deadline)的数据流在带宽分配和占用上应该被区分开,
网络负载越重,数据流越应该被区分。
根据这个原则,本文提出了一种简单的拥塞控制算法-正比负载差分策略
(Load Proportional Differentiation,简称LPD)作为其应用。
并在不同的拓扑和负载情况下评估了LPD。


\textbf{3 提出了基于流持续时间的传输优化方案,同时可以优化有期限的流以及流完成时间}

本文主张使引入流持续时间到拥塞窗调整的过程中。
基于此,本文提出流持续时间速率控制机制(Flow Duration Time Rate Control,简称 FDRC)。
在不用预先得知流信息的情形下,FDRC 可以减 少流错失期限的比例并且能够减小流平均完成时间。
本文从理论上分析了 FDRC 的行为,并在 ns-2 和 Iinux 内核实现 FDRC,并对之进行评估。



\textbf{4 提出了数据中心任务级传输优化模型,并提出数据中心流组调度的离线调度策略}


本文对数据中心任务传输模型进行介绍,并且从数据中心非阻塞模型,以及传输任务等对数据中心任务级传输的模型进行介绍。
本文推算出数据中心任务调度问题的复杂度,引入任务的离线调度算法,并证明离线调度算法的近似度。


\textbf{5 提出了基于流信息的任务级传输优化方案,可以在预习得知流大小前提下优化任务传输}

本文提出基于流信息的任务级传输优化方法来最小化文件平均访问时间(File Access Time,简称FAT)。
本文同时结合了最小负载优先的启发式算法做为从N个数据块中选取K个数据块的方法。
在此基础上,本文设计并实现了D-Target,一个集中式调度器,对文件系统的源选择以及文件系统的任务传输进行整体优化。


\textbf{6 提出了基于重要性和网络拥塞的任务传输调度方案,可以无需预先得知流大小即能优化任务传输时间}


本文引入加权流组完成时间(Weighted Coflow Completion Time,简称WCCT)问题,
设计一个不需要知道流组(coflow)信息就可以进行有效调度的算法,根据权重和网络拥塞程度调度流组(coflow),
同时设计一个名为Yosemite的调度算法,并通过数据中心的真实流量对Yosemite进行仿真评测。
各方案对比,如表\ref{compare}所示。

\begin{table}[h]
\centering
\caption{流级别优化方案}\label{compare}
\renewcommand{\arraystretch}{1.5}
\begin{tabular}{|c|c|c|c|c|c|c|c|} \hline
\setlength{\tabcolsep}{10pt}
策略& 粒度 & 方法&期限&任务完成时间&预先得知流大小&网络拥塞和优化目标关系\\ \hline
LPD &流级别&分布式&是&否&是& 是\\ \hline
FDRC &流级别&分布式&是&是&否& 是\\ \hline
D-Target &任务级别&集中式&否&是&是& 是\\ \hline
Yosemite &任务级别&集中式&否&是&否& 是\\ \hline
\end{tabular}
\end{table}






\textbf{7 设计并且实现了数据中心传输系统,实现了数据中心应用流和任务的传输优化}


本文设计并实现了数据中心传输系统FlyTransfer,一个基于集中式的调度系统,
FlyTransfer可以对数据中心的应用实现流级别和任务级别的传输优化和调度。
最后,本文将FlyTranfer部署在openstack平台上,并对之进行性能评估。


\section{进一步研究工作}
进一步研究工作包括以下今个方面

\textbf{1 虚拟环境下流传输优化}

最近,随着SDN等技术的兴起,虚拟化网络技术用途越来越广泛,许多应用开始部署在虚拟环境中。
最近的一些研究\cite{He2016AC,Cronkite2016Virtualized}在虚拟化环境从传输层进行优化。
VCC\cite{Cronkite2016Virtualized},在hypervisor上进行优化,
解决了虚拟化和物理环境下基于ECN标记的方法和非ECN标记方法的不公平问题。
AD/DC\cite{He2016AC}在虚拟环境下实现DCTCP。
然后,到目前为止,还没有对虚拟环境下基于期限进行传输优化的方法,
因此,在虚拟环境下根据网络拥塞情形以及CPU使用率,应用特性进行调度是未来研究的热点问题。



\textbf{2 虚拟环境下任务传输优化}

随着Docker等容器技术的发展和成型,诸如Hadoop,Spark等分布式应用开始部署在容器上。
通物理传输相同,仅仅流级别传输是不够的,需要进行任务级的传输优化。
仅仅将物理平台上的传输方案平移到虚拟平台上是不够的,因为许多容器共享物理机的CPU,这会导致传输震荡大。
因此,需要结合任务放置或者CPU使用率方面进行传输优化。当前,还未出现虚拟环境下任务传输优化系统和平台。


\textbf{3 分布式任务传输}

无论是Varys还是Aalo,都需要借助集中处理器进行或多或少的计算。
集中处理的方式可以根据网络和应用的整体情况进行调度。
集中式的方法计算准确,但是存在两个不足。
首先,集中式方式会增加延迟。采取集中控制的方法,节点需要将信息发送给中央控制器,
然后中央控制器进行计算,随后把结果反馈给节点,这个过程会增大延迟。
其次,集中控制的方式,控制器容易成为系统瓶颈。集中控制的方法,中央控制器存储着网络所有节点的信息,如果中央控制器发生故障或者瘫痪,
会使网络整体进入瘫痪状态。
因此,分布式的任务传输平台在鲁棒性以及性能上会更有优势。当前业界还没有出现类似解决方案。




\textbf{4 结合路径选择的任务传输优化}

当前数据中心对于任务的传输优化,大多假设数据中心是非阻塞结构,
在此结构下,拥塞只发生在数据传输的第一跳和最后一跳。
事实上,在数据中心,拥塞不仅发生在第一跳和最后一跳,
中间路径也会发生拥塞。因此,拓展非阻塞模型对任务进行传输优化是未来研究的重点内容。


