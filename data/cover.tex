\thusetup{
  %******************************
  % 注意:
  %   1. 配置里面不要出现空行
  %   2. 不需要的配置信息可以删除
  %******************************
  %
  %=====
  % 秘级
  %=====
  secretlevel={秘密},
  secretyear={10},
  %
  %=========
  % 中文信息
  %=========
  ctitle={数据中心网络中的应用吞吐率和延迟优化},
  cdegree={工学博士},
  cdepartment={计算机科学与技术系},
  cmajor={计算机科学与技术},
  cauthor={张晗},
  csupervisor={尹霞教授},
  %cassosupervisor={陈文光教授}, % 副指导老师
  %ccosupervisor={某某某教授}, % 联合指导老师
  % 日期自动使用当前时间,若需指定按如下方式修改:
  % cdate={超新星纪元},
  %
  % 博士后专有部分
%  cfirstdiscipline={计算机科学与技术},
%  cseconddiscipline={系统结构},
%  postdoctordate={2009年7月——2011年7月},
%  id={编号}, % 可以留空: id={},
%  udc={UDC}, % 可以留空
%  catalognumber={分类号}, % 可以留空
  %
  %=========
  % 英文信息
  %=========
  etitle={Throughput and latency optimization for applications in data center networks},
  % 这块比较复杂,需要分情况讨论:
  % 1. 学术型硕士
  %    edegree:必须为Master of Arts或Master of Science(注意大小写)
  %             “哲学、文学、历史学、法学、教育学、艺术学门类,公共管理学科
  %              填写Master of Arts,其它填写Master of Science”
  %    emajor:“获得一级学科授权的学科填写一级学科名称,其它填写二级学科名称”
  % 2. 专业型硕士
  %    edegree:“填写专业学位英文名称全称”
  %    emajor:“工程硕士填写工程领域,其它专业学位不填写此项”
  % 3. 学术型博士
  %    edegree:Doctor of Philosophy(注意大小写)
  %    emajor:“获得一级学科授权的学科填写一级学科名称,其它填写二级学科名称”
  % 4. 专业型博士
  %    edegree:“填写专业学位英文名称全称”
  %    emajor:不填写此项
  edegree={Doctor of Engineering},
  emajor={Computer Science and Technology},
  eauthor={Han Zhang},
  esupervisor={Professor Xia Yin},
  % 日期自动生成,若需指定按如下方式修改:
  % edate={December, 2005}
  %
  % 关键词用“英文逗号”分割
  ckeywords={数据中心, 网络, 应用, 延迟, 带宽},
  ekeywords={Date Center,network,application,latency,throughput}
}

% 定义中英文摘要和关键字
\begin{cabstract}
在过去的几年中,越来越多的企业有了自己的数据中心,还出现了比如Amazon,微软和谷歌这类数据中心服务提供商。
在数据中心如何使用廉价、常见的网络设备来给应用提供低延迟、高带宽服务是十分重要的问题。
TCP是互联网中应用最多的传输协议。
尽管TCP被广泛的应用在数据中心,但是在数据中心中使用TCP存在以下问题:
(1)TCP使用基于丢包的拥塞检测方法,当网络中出现拥塞时,数据包被丢弃,导致发送端发送速率震荡大,网络利用率低。
(2)交换机的队列长,排队延迟高。
(3)应用的数据流均分网络带宽,不能满足不同应用对带宽和延迟不同的需求。
(4)TCP是数据流级别的传输,难以满足数据中心传输任务对时延的要求。

因此,需要对数据中心应用传输同时进行流级别和任务级别的传输优化,并兼顾网络拥塞和应用特征。
基于此,本文从流和任务级别对数据中心中的传输和调度方法做出优化,主要研究内容和贡献如下:


(1)提出了数据中心传输模型,具体包括了基于ECN的流传输模型和任务级别传输优化模型。
通过基于ECN的流传输模型,可以推导出基于ECN流传输协议的参数设置区间。
针对数据中心任务传输,本文提出理想化权重流组完成时间优化 (Idealized Weighted Coflow Completion Time Minimization,简称 IWCCTM) 问题,
并提出2-近似离线算法解决此问题。


(2)针对流级别的优化,本文提出了期限自适应的流传输算法-LPD和基于流传输时间的速率控制机制-FDRC。
其中,LPD针对基于期限的拥塞控制机制在网络负载严重时失效的问题,
提出数据中心基于期限的控制策略应该遵循“越拥塞,越区分”的设计原则,
使的网络拥塞严重时,依然可以根据期限进行速率控制。
FDRC针对数据中心内存在有期限的流和短流,当前机制并不能同时满足这两种流需求的问题,提出基于流持续时间的拥塞控制机制,
使的错失期限的流和流平均完成时间都大幅减少。

    
(3)针对任务级别的优化,本文提出基于流信息的调度策略D-Target和基于任务重要性及网络拥塞的调度策略Yosemite。
其中D-Target针对网络中使用随机选源和流级别传输导致文件访问延迟较大的问题,
提出在预先得知流信息的前提下,结合纠删码存储系统的源选择,优化文件平均存取时间。
Yosemite针对未考虑应用重要性,导致重要任务传输性能低下的问题,提出可以不预知流信息的前提下,优化平均权重任务传输时间的策略。

(4)设计并且实现了传输优化系统FlyTransfer,并在OpenStack等真实环境下进行了部署和测试。
实验发现,FlyTransfer可以同时进行流级别和任务级别的优化。
本文还对FlyTransfer进行了性能评估。
  
\end{cabstract}

% 如果习惯关键字跟在摘要文字后面,可以用直接命令来设置,如下:
% \ckeywords{\TeX, \LaTeX, CJK, 模板, 论文}
\begin{eabstract}
In the past few years, more and more companies have their own data centers.
Some companies such as Amazon, Microsoft, and Google even provide data center service. 
In this case, how to use cheap, common network devices to provide low-latency, high-bandwidth services to the applications of data center is an important problem. TCP is the most widely used transport protocol in the Internet. 
Although TCP is widely used in the data center, 
the traditional TCP transport protocol has following problems:
 (1) TCP's congestion window will be halved when there is congestion in the network
 and this will lead the network utility at low level.
 (2) Queue delay of TCP is too long to satisfy the latency demands of applications. 
 (3) TCP is a fair allocation method and fair allocation can not meet the demands of applications on bandwidth and latency. 
 (4) TCP can not meet the requirements of task level transmission.
 
As a result, it is necessary to optimize the transport of applications from flow level and task level.
Also, the different bandwidth and latency demands of applications should also be taken into consideration.
Based on these, we make the following contributions in this paper:
 
(1) We propose flow transfer model and task optimization model. 
With the ECN-based flow level model, parameters of rate control methods can be analyzed.
Based on the task-level model, we propose Idealized Weighted Coflow Completion Time Minimization(IWCCTM) problem.
To solve IWCCTM problem, we propose a 2-approximate optimal method to solve.
  
(2) For the optimization of flow level, 
we propose the load adaption protocol-LPD and the flow duration time rate control protocol-FDRC.
As most policies lose efficacy under heavy congestion,
LPD advocates that both network condition and deadline should be taken when designing rate control algorithm.
Then following the principle "more load, more differentiation", LPD works well even when the network is under heavy load.
Due to the traffic is the mixture of deadline sensitive and short flows, modern methods can not meet the requests of the two kind flows simultaneously.
FDRC proposes to use flow duration time when designing congestion control methods.
As a result, both the percentage of flows missing deadline and flow completion time reduce.

(3) For the optimization of task level,
we propose the centralized schedule methods D-Target and Yosemite.
With the information of flows and for the problem of TCP transportation as well as random source selection,
D-Target can reduce File Access Time (FAT) significantly.
As scheduling methods ignore the importance of coflows, where important coflows have poor performance.
Yosemite considers both the network condition and the importance of coflows, as a result,
the performance of important coflows improves.
 

(4) We design and evaluate FlyTransfer, a system that can schedule the transfer of flows as well as tasks. 
We deploy FlyTransfer at the platform of openstack and other real world platform.
Evaluations show that FlyTransfer can optimize the transfer of tasks and flows with small overheads.
We also test the overhead of FlyTransfer in this paper.
  
  
\end{eabstract}

% \ekeywords{\TeX, \LaTeX, CJK, template, thesis}
